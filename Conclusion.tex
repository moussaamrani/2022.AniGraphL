\section{Conclusion}
\label{sec:Conclusion}

Model Animation (MA) is a popular approach for providing visual clues for model transformation
designers, as well as modellers, that should contribute to help them better 
understand, and ultimately correct, their models and the associated transformations.
This is because MA is closely related to the execution of \DSLs: MA visually represent
meaningful steps of the model transformations. 

As a first step towards a systematic approach for MA engineering, we have 
identified in this paper three important challenges. 
First, designing concrete syntaxes for \DSLs is not trivial, and this should support
complex patterns in order to explicitly express the so-called \emph{mapping}, i.e.
the rendering transformation between the \DSL metamodel capturing its abstract syntax,
into the metamodel describing graphical components. This latter metamodel is a \DSL
on its own right, and should natively support animation, i.e. the timely and 
effective change in the graphical components's features (e.g. size, color, thickness,
and general topology of elements such as tables, rectangles, etc.)

Second, we plea for a compositional approach for defining complex MA, because relying
on basic constructions that are combined to form complex animations promotes flexibility
and reuse. The flexibility of a dedicated MA \DSL should open the ability to 
parameterise MA definitions, so that the internal logic of the MA becomes independent
from the graphical elements it is applied to. Just like methods may be used in 
different context captured by the method's parameters, an MA should have parameters
to explicitly designate which elements are taken into account, and how. The reuse
of an MA \DSL should allow to apply the same MA, which uses the same logic underneath,
across multiple \DSLs that have the same animation logic. We have shown on simple
examples that both properties, flexibility and reuse, are effectively found in
very simple \DSLs, and should appear in bigger ones as well. In turn, these properties
should promote the exchange and reuse of well-defined libraries of animations, thus
facilitating the specification of future MAs.

Third, in order to not reinvent the logic behind the Model Transformation capturing
a \DSL's behavioural semantics, an MA should be linked to appropriate Transformations
Units (TUs) that would drive the MAs. This way, it becomes possible to extensively
reuse the Model Transformation specification as well as its scheduling logic, and
only trigger appropriate MA pieces (that we called Animation Units) that would
visually reflect what a TU is achieving. 

We are aware that MA is a novel field, thus lacks maturity both in terms of methodologies
and tooling. However, by identifying these challenges, by providing some leads on
how it could be possible to at least partially meet them, while still applying
the \MDE approach (and in particular, defining carefully designed \DSLs for each
of the above tasks), we hope to pave the way towards a new and systematic approach
for MA. This, in turn, should create a new role for \MDE: MA designer, alongside
the usual role of modeller, \DSL builder and MT specialist. As the field is still
in its infancy, we are planning to approach the literature more systematically,
in order to uncover what the current practice is, how it may be possible to 
formalise the \DSLs dedicated to the tasks we identified, and identify the various
features of animators as a tool in the collection of tools for \MDE workbenches.