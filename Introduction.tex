\section{Introduction}
\label{sec:Introduction}

Investing the time and efforts to develop the tooling associated with general-purpose
programming or modelling langages, such as analysers, debuggers, and testing
frameworks, is a no-brainer, because of the large audience it will benefit.
The same question for Domain-Specific (Modelling) Languages (\DSMLs) has a
more mitigated answer. For example, in recent years, many efforts have been
invested in defining guidelines for the systematic design and development of 
\DSL debuggers 
\cite{bousse2018omniscient,J:VanMierlo-Vangheluwe-etAl:2020,J:Corley-Eddy-Syriani-Grey:2016},
but also various analysis tools \cite{Meyers-Deshayes-etAl:2014}, resulting in
tool support that automate most part of the development. This helps reduce the
time and efforts \DSL tool builders have to invest to offer, alongside of their
\DSL, supporting tools that are nowadays considered essential for the activity
of \DSL modelling.

A complementary, lightweight approach for ensuring correctness and quality of
\DSLs, in particular concerning their behaviour, is the adoption of Model 
Animation (MA). MA can be seen as the visual representation of the simulation of
Model Transformations (MTs) \cite{J:Lucio-Amrani-etAl:2014}: it gives a visual 
counterpart to the changes and updates operated on a model during its execution.
Just like debugging, MA assumes that an MT captures the behavioural semantics 
of a \DSL, namely an inplace, endogenous \emph{simulation}, and that the MT 
engine is capable of interrupting its execution in key places so that the 
modeller can inspect visually what is happening. However,
since MA heavily relies on model visualisation, another crucial element is necessary:
the definition of a \emph{visual} concrete syntax for the model, and an associated
MA engine that has the ability to alter graphical features of the concrete syntax
fast enough to confer the illusion of movement, hence the animation. Relying
on visual information, MA provides an insight and feedback about what is going on
during a simulation that would help modellers to understand, analyse, and predict
what their models do by relying on visual information, especially those who have
little background in programming. It should also support MT designers for understanding,
tracing, monitoring, and ultimately debugging their specifications based on visual
clues.

Many \DSL tools already offer the ability to perform MA alongside of MT. eProvide,
\citep{Sadilek-Wachsmuth:2008} (now discontinued) was one of the first tools that
included a ``visual debugger'', which required to alter the \DSL metamodel to 
accommodate with the debugging logic, and relied on simple graphical components.
More recently, GeMoC \citep{combemale2016tool} and AtoMPM \cite{Syriani-Vangheluwe-etAl:2013}
are two well-established \DSL frameworks that both integrate debugging and MA. 
The animators are tightly coupled with the underlying MT Language (MTL) for 
performing animation (although GeMoC offers several ones, namely Kermeta and Xtend), since the
execution engine is instrumented to stop at specific locations in the MT to perform
animation pieces. An interesting feature of AtoMPM is the ability to take advantage
of the concrete syntax to express the MT rules themselves, reducing the gap between
MT specification and MA. Other tools based on formalisms that have native graphical
representations (such as Finite State Machines and Petri Nets) offer animators
with little extra efforts since the MT is directly expressed using the visual MT,
whose execution is simply rendered visually. However, these tools force MT designers to 
specify their transformations in a formalism that does not directly manipulate their
metamodel's features. 

Although these tools all represent interesting steps and real progress in the matter
of MA, they sometimes lack \emph{flexibility} in the MA specification, and 
\emph{reuse} of MA units across \DSLs. 
In this paper, we take a first step towards the systematic engineering of MA, by
identifying key components for designing MA tools, such that they ensure our
required properties of flexibility and reuse. We formulate three challenges that
touch upon the core components of animators: the concrete syntax and the associated
MA engine; the MA specification; and the relationship between MA and MT.
We detail and elaborate these challenges by showing on simple examples what we
intend by flexibility and reuse, and how a compositional \DSL dedicated to animation
can help with these requirements. 

The paper is organised as follows. We start in \S \ref{sec:Motivation} by clarifying
the notion of MA in contrast to closely related notions, and motivate our challenges
with concrete situations. We review in \S \ref{sec:Examples} some popular \DSLs,
and devise a family of animation for each of them, showing that animation should
be left as a design choice, instead of being forced by an execution engine. 
We then revisit the challenges in \S \ref{sec:Challenges}, to identify key 
ingredients for each of them that would support the ability to build animators.
To conceptually meet the challenges, we formulate a (partial) proposal in \S 
\ref{sec:Proposal} and show on our example \DSLs how some of the challenges find
an interesting answer. Finally, we discuss Related Work in \S \ref{sec:RW} before
wrapping up in \S \ref{sec:Conclusion} with concluding remarks.
