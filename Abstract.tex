\begin{abstract}
Model Animation (MA) is a practical technique for providing modellers and Model
Transformation designers an insurance that their models behave as expected.
This is specially relevant in expertise domains where models have a natural
visual representation, i.e. a dedicated concrete syntax. 
MA can be seen as the visual representation of Model Transformation simulation. 
It supports Model Transformation designers understand, trace,
monitor, and ultimately debug their specification using visual clues.

In contrast to other techniques surrounding Model-Driven Engineering, MA has received
less attention, in contrast to, e.g. testing or debugging.
This paper is a first step towards the systematic engineering of model animators.
It identifies three key challenges: (i) how to effectively, explicitly and precisely
define the concrete syntax to enable MA; (ii) how to build an MA language to express
animations units in a compositional way, so that animations become flexible in their
definition, and reusable across several \DSLs; and finally (iii) how to explicitly
relate MA units with their transformation counterparts, to avoid reimplementing the
transformation scheduling. 

We analyse these challenges to extract some requirements for future animators,
and give a partial conceptual proposal that fulfil them, paving the way towards
the creation of a family of animation tools that would work alongside transformation
engines. We then show, on simple examples, how these propositions apply and to which
extent they promote flexibility and reuse.
\end{abstract}
